\chapter{Feature Engineering und Selektion}

% TODO: Wird nach Phasen 02-04 geschrieben
% Dokumentiert Multikollinearitätsanalyse und Feature Selection

\section{Explorative Datenanalyse}

\subsection{Univariate Analyse}
% - Verteilungen je Feature
% - Unterschiede zwischen Insolvent/Nicht-Insolvent
% - Horizontspezifische Unterschiede

\subsection{Bivariate Analyse}
% - Korrelationsmatrizen
% - Scatter Plots für Top-Features
% - Identifikation nichtlinearer Zusammenhänge


\section{Multikollinearitätsanalyse}

\subsection{Variance Inflation Factor (VIF)}
% - Methode: VIF-Berechnung für alle 64 Features
% - Schwellenwert: VIF > 10
% - Erwartung: Viele Features betroffen (inverse Paare, gemeinsame Nenner)

\subsection{Korrelationsbasierte Filterung}
% - Pearson-Korrelation zwischen Features
% - Entfernung bei |r| > 0.95
% - Priorisierung: Ökonomische Interpretierbarkeit


\section{Feature Selection}

\subsection{Filter-Methoden}
% - Chi-Quadrat-Test für kategoriale Beziehungen
% - ANOVA F-Test für kontinuierliche Features
% - Mutual Information Score

\subsection{Wrapper-Methoden}
% - Forward Selection
% - Backward Elimination
% - Recursive Feature Elimination (RFE)

\subsection{Embedded-Methoden}
% - Feature Importance aus Random Forest
% - Permutation Importance
% - SHAP Values für Interpretierbarkeit


\section{Finale Feature-Sets}

\subsection{Reduzierte Feature-Sets je Horizont}
% - H1: X Features (nach VIF + Selection)
% - H2: Y Features
% - etc.

\subsection{Validierung der Feature-Sets}
% - Cross-Validation Performance
% - Stabilität über Folds
% - Ökonomische Plausibilität
