\chapter{Modellierung und Evaluation}

% TODO: Wird nach Phase 05 geschrieben
% Dokumentiert alle ML-Modelle und deren Evaluation

\section{Modellarchitekturen}

\subsection{Baseline: Logistische Regression}
% - GLM mit Logit-Link (NICHT OLS!)
% - Regularisierung: L2 (Ridge)
% - Hyperparameter: C (Inverse Regularisierungsstärke)
% - Begründung: Interpretierbare Koeffizienten

\subsection{Random Forest}
% - Ensemble-Methode: Bagging + Random Feature Selection
% - Hyperparameter: n\_estimators, max\_depth, min\_samples\_split
% - Vorteil: Robust gegen Multikollinearität
% - Feature Importance verfügbar

\subsection{XGBoost}
% - Gradient Boosting mit Regularisierung
% - Hyperparameter: learning\_rate, n\_estimators, max\_depth, subsample
% - Vorteil: State-of-the-art Performance
% - Frühe Stoppung zur Vermeidung von Overfitting


\section{Hyperparameter-Tuning}

\subsection{Grid Search mit Cross-Validation}
% - 5-Fold Stratified CV
% - Suchraum je Modell definiert
% - Optimierung auf ROC-AUC

\subsection{Horizontspezifisches Tuning}
% - Jeder Horizont hat eigene optimale Parameter
% - Dokumentation aller gewählten Werte


\section{Behandlung der Klassenimbalance}

\subsection{Class Weights}
% - Inverse Häufigkeit als Gewicht
% - Implementierung in allen Modellen
% - Alternative: SMOTE (synthetische Samples)

\subsection{Threshold-Optimierung}
% - Standard: 0.5
% - Optimierung auf F1-Score oder Precision-Recall
% - Horizontspezifische Schwellenwerte


\section{Evaluation Metrics}

\subsection{Klassifikations-Metriken}
% - Accuracy (wenig aussagekräftig bei Imbalance!)
% - Precision, Recall, F1-Score
% - ROC-AUC (Hauptmetrik)
% - Precision-Recall AUC

\subsection{Kalibrierung}
% - Calibration Plots (Reliability Diagrams)
% - Brier Score
% - Platt Scaling falls nötig

\subsection{Konfusionsmatrizen}
% - True Positives, False Positives
% - True Negatives, False Negatives
% - Ökonomische Interpretation (Typ I vs Typ II Fehler)


\section{Modellvergleich}

\subsection{Performance je Horizont}
% - Tabelle: Modell × Horizont × Metrik
% - Bester Modelltyp je Horizont
% - Statistische Signifikanz (McNemar Test)

\subsection{Feature Importance Analyse}
% - Top 10 Features je Modell und Horizont
% - Konsistenz über Modelle
% - Ökonomische Interpretierbarkeit

\subsection{SHAP Values für Interpretierbarkeit}
% - Shapley Additive Explanations
% - Global Importance
% - Lokale Erklärungen (Einzelvorhersagen)
